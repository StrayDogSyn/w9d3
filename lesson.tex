% Created 2025-06-11 Wed 16:07
% Intended LaTeX compiler: pdflatex
\documentclass[aspectratio=169]{beamer}
\usepackage[utf8]{inputenc}
\usepackage[T1]{fontenc}
\usepackage{graphicx}
\usepackage{longtable}
\usepackage{wrapfig}
\usepackage{rotating}
\usepackage[normalem]{ulem}
\usepackage{amsmath}
\usepackage{amssymb}
\usepackage{capt-of}
\usepackage{hyperref}
\usepackage[style=apa, backend=biber]{biblatex}
\DeclareLanguageMapping{american}{american-apa}
\addbibresource{./refs/refs.bib}
\AtEveryBibitem{\clearfield{note}}
\usepackage{./jtc}
\usetheme{default}
\author{Evan Misshula}
\date{\today}
\title{Introduction to Machine Learning: Evaluation and Training Math Review}
\hypersetup{
 pdfauthor={Evan Misshula},
 pdftitle={Introduction to Machine Learning: Evaluation and Training Math Review},
 pdfkeywords={},
 pdfsubject={},
 pdfcreator={Emacs 29.3 (Org mode 9.6.15)}, 
 pdflang={English}}
\begin{document}

\maketitle

\section{What Is a Confusion Matrix?}
\label{sec:org27ef0b7}
\begin{frame}[label={sec:orgfbb8cd0}]{Definition}
\begin{itemize}
\item A \alert{confusion matrix} is a table that summarizes the performance of a classification model.
\item It compares the predicted labels with the actual labels.
\item Especially useful for binary or multiclass classification.
\end{itemize}
\end{frame}

\begin{frame}[label={sec:org384791a}]{Binary Confusion Matrix}
\begin{center}
\begin{tabular}{lll}
 & Predicted Positive & Predicted Negative\\[0pt]
\hline
Actual Positive & True Positive (TP) & False Negative (FN)\\[0pt]
Actual Negative & False Positive (FP) & True Negative (TN)\\[0pt]
\end{tabular}
\end{center}
\end{frame}

\begin{frame}[label={sec:org0912c6d}]{Notation}
Let:
\begin{itemize}
\item \(y_i \in \{0, 1\}\) be the true label
\item \(\hat{y}_i \in \{0, 1\}\) be the predicted label
\end{itemize}

Then:
\begin{itemize}
\item TP: \(\sum \mathds{1}_{\{y_i = 1 \wedge \hat{y}_i = 1\}}\)
\item TN: \(\sum \mathds{1}_{\{y_i = 0 \wedge \hat{y}_i = 0\}}\)
\item FP: \(\sum \mathds{1}_{\{y_i = 0 \wedge \hat{y}_i = 1\}}\)
\item FN: \(\sum \mathds{1}_{\{y_i = 1 \wedge \hat{y}_i = 0\}}\)
\end{itemize}
\end{frame}

\section{Performance Metrics}
\label{sec:orge876c59}
\begin{frame}[label={sec:org62489d8}]{Accuracy}
\[
\text{Accuracy} = \frac{TP + TN}{TP + TN + FP + FN}
\]
\begin{itemize}
\item Proportion of total predictions that were correct.
\item Best used when classes are balanced.
\end{itemize}
\end{frame}

\begin{frame}[label={sec:org23cdd64}]{Precision}
\[
\text{Precision} = \frac{TP}{TP + FP}
\]
\begin{itemize}
\item Among predicted positives, how many were actually positive?
\end{itemize}
\end{frame}

\begin{frame}[label={sec:org426437c}]{Recall (Sensitivity, TPR)}
\[
\text{Recall} = \frac{TP}{TP + FN}
\]
\begin{itemize}
\item Among actual positives, how many did we correctly predict?
\end{itemize}
\end{frame}

\begin{frame}[label={sec:org13cb68e}]{F1 Score}
\[
F1 = 2 \cdot \frac{\text{Precision} \cdot \text{Recall}}{\text{Precision} + \text{Recall}}
\]
\begin{itemize}
\item Harmonic mean of precision and recall.
\item Balances false positives and false negatives.
\end{itemize}
\end{frame}

\begin{frame}[label={sec:org163d67c}]{Specificity}
\[
\text{Specificity} = \frac{TN}{TN + FP}
\]
\begin{itemize}
\item Among actual negatives, how many did we correctly predict?
\end{itemize}
\end{frame}

\section{Multiclass Confusion Matrix}
\label{sec:orgcdee63a}
\begin{frame}[label={sec:org69dcda3}]{Generalization}
\begin{itemize}
\item For \(K\) classes, the confusion matrix is \(K \times K\)
\item Entry \((i,j)\) is the number of times class \(i\) was predicted as class \(j\)
\end{itemize}
\end{frame}

\begin{frame}[label={sec:orgd5f1001}]{Use Cases}
\begin{itemize}
\item Visualizing classifier performance
\item Identifying types of errors
\item Computing per-class precision and recall
\end{itemize}
\end{frame}


\begin{frame}[label={sec:orge3fcd83}]{Key Takeaways for the Confusion Matrix}
\begin{itemize}
\item Confusion matrix is foundational for classifier evaluation.
\item Metrics derived from it (precision, recall, F1) offer deeper insight than accuracy alone.
\item Always inspect confusion matrices — especially on imbalanced datasets.
\end{itemize}
\end{frame}

\section{ROC Curve}
\label{sec:orge351a26}
\begin{frame}[label={sec:orgf6cd047}]{Motivation}
\begin{itemize}
\item ROC AUC is a standard metric for evaluating binary classifiers.
\item Focuses on ranking predictions rather than absolute accuracy.
\item Especially useful with imbalanced data or when decision thresholds vary.
\end{itemize}
\end{frame}


\begin{frame}[label={sec:org5db1ab0}]{What Is the ROC Curve?}
\end{frame}
\section{ROC Curve}
\label{sec:orgd02bce3}
\begin{frame}[label={sec:orgf63fc57}]{What Is the ROC Curve?}
\begin{itemize}
\item \alert{Receiver Operating Characteristic (ROC) curve}:
\begin{itemize}
\item A graphical plot that shows the trade-off between \alert{True Positive Rate} (TPR) and \alert{False Positive Rate} (FPR).
\end{itemize}
\item The curve is constructed by \alert{sweeping a decision threshold} over the predicted probabilities output by the model.
\end{itemize}
\end{frame}

\begin{frame}[label={sec:org4fb32d4}]{Understanding the Threshold}
\begin{itemize}
\item Most classifiers (like logistic regression) output a probability score \(\hat{p} \in [0, 1]\).
\item We need to decide: \alert{at what probability value do we say "yes, this is a positive"?}
\item This cut-off value is called the \alert{threshold}.
\end{itemize}
\end{frame}

\begin{frame}[label={sec:orgd3db826}]{Example:}
\begin{itemize}
\item If threshold = 0.5:
\begin{itemize}
\item \(\hat{p} \geq 0.5 \Rightarrow\) predict \alert{positive}
\item \(\hat{p} < 0.5 \Rightarrow\) predict \alert{negative}
\end{itemize}
\item Lowering the threshold means more predictions are labeled positive, increasing TPR but also increasing FPR.
\item Raising the threshold means fewer predictions are labeled positive, which may reduce FPR but also lower TPR.
\end{itemize}
\end{frame}

\begin{frame}[label={sec:orgf66081b}]{Each point on the ROC curve corresponds to:}
\begin{itemize}
\item A different threshold
\item A pair \((\text{FPR}, \text{TPR})\) computed using that threshold
\item Sweeping the threshold from 0 to 1 traces out the entire ROC curve
\end{itemize}
\end{frame}
\end{document}
